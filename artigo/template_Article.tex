\documentclass[12pt,openright,oneside,a4paper,brazil]{abntex2}


\usepackage{amsmath}
\usepackage{amsmath,amsthm,amsfonts,amssymb,amscd}
\usepackage{graphicx}
\usepackage[ruled,vlined]{algorithm2e}
\graphicspath{{./grafos/}{./}} % where to search for the images

%opening
\titulo{Trabalho de Analise de Algoritmos}
\autor{Felipe Augusto Ferreira de Castro \textbf{Matrícula:} 11711BCC033\\Sarah Hanna VB  Silva \textbf{Matrícula:} 11621BCC021\\ Renata Cristina Gomes da Silva \textbf{Matrícula:} 11721BCC012}
\data{2021}
\local{Universidade Federal de Uberlândia}
\begin{document}

\imprimircapa
\newpage
\tableofcontents

\chapter{Coloração de Grafos}

\section{Pseudo-código}

O problema de coloração de grafos é um problema bastante discutido na literatura da área e possui vários algoritmos para soluciona-lo. Portanto, nesta seção apresentaremos o algoritmo usado em nosso trabalho para encontrar a disposição de cores e posteriormente discutiremos a eficiência do algoritmo. A seguir o algoritmo utilizado apresentado em pseudo-código.\linebreak

\begin{algorithm}[H]
\SetAlgoLined
\KwIn{grafo}
\KwOut{Lista com as cores de cada vertice}
declare uma variavel flag;

faça flag = verdadeiro;

atribua a cor 0 para todos os vertices do grafo\;

primeiro vertice do grafo\; 

\While{não atribuido uma cor a todos os vertices}{	
	
	\While{para todos os vertices adjacentes }
	{ 
		olhe a cor do vertice adjacente\;
		
		\eIf{possui a mesma cor que o vertice adjacente}
		{
			
		 faça flag = falso\;
		 	
		 pare o laço\;		
		}{}
	}
	\eIf{flag =  verdadeiro}{
		proximo vertice;
		
	}{  
		faça flag = verdadeiro\;
	    some 1 a cor deste vertice\;
	}
}
\caption{Coloraçao de Grafos}
\end{algorithm}

Explicando de maneira mais informal o algoritmo se resume em alguns passos:
\begin{itemize}
	\item Para cada vértice $v$ do grafo $G$ vamos olhar as cores de seus vertices adjacentes;
	\item vamos avançando da lista de cores ate encontrar uma cor a qual não foi atribuida a nenhum vértice adjacente a $v$;
	\item atribuímos a cor encontrada ao vertice $v$; 
\end{itemize}

\section{Estrutura de dados}
Visto o algoritmo para coloração apresentado na seção anterior foi decidido estruturar o grafo de maneira a facilitar encontrar os vértices adjacentes de cada vértice $v$  do grafo $G$. Desta forma, cada vértice é uma estrutura que possui duas informações:
\begin{itemize}
	\item valor da cor atribuída ao vértice;
	\item uma lista de identificadores dos vértices adjacentes;
\end{itemize}

A identificação do vértice adjacente $vd$ é feita com um indexador da posição de $vd$ na lista de vértices do grafo $G$.

A Estrutura do grafo é constituída de duas informações: 
\begin{itemize}
	\item Lista de vértices presentes no grafo(o indexador do vértice nessa lista é o identificador do vértice);
	\item quantidade de arestas presentes no grafo;
\end{itemize}

Utilizando desta estrutura encontrar os vértices adjacentes de um vértice $v$ se resume a apenas percorrer uma lista, excluída a necessidade de verificar se o vértice $vd$ é adjacente a $v$.

\section{Desenvolvimento do Trabalho}
O programa foi desenvolvido na linguagem Python devido a facilidade de encontrar ferramentas prontas para manipular estruturas de dados e o conhecimento prévio que os autores deste trabalho tinham sobre a linguagem. Além disso, o ambiente de desenvolvimento  usado foi o Visual Studio Code.


\subsection{Grafos}   
Quantos aos grafos, foram utilizados 5 grafos, no formato DIMACS, os quais serão apresentados a seguir. 

\begin{enumerate}
	\item 	\textbf{Grafo 1} \newline
		vertices: 10 , arestas: 15,
		a 0 1, \newline
		a 0 2, \newline
		a 0 3, \newline
		a 1 4, \newline
		a 1 8, \newline
		a 2 6, \newline
		a 2 7, \newline
		a 3 5, \newline
		a 3 9, \newline
		a 4 5, \newline
		a 4 7, \newline
		a 5 6, \newline
		a 6 8, \newline
		a 7 9, \newline
		a 8 9;
		
		Fonte: O enunciado deste trabalho
	\item \textbf{Grafo 2} \newline 
		vertices: 9, arestas: 14, \newline
		a 0 3, \newline
		a 0 1, \newline
		a 1 2, \newline
		a 1 3, \newline
		a 2 4, \newline
		a 3 4, \newline
		a 3 6, \newline
		a 3 7, \newline
		a 4 5, \newline
		a 4 7, \newline
		a 4 8, \newline
		a 5 8, \newline
		a 6 7, \newline
		a 7 8;
		
		Fonte: https://coloringbee.blogspot.com/2018/09/coloring-graph-example.html
\end{enumerate}
\end{document}
