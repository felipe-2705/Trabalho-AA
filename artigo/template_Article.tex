\documentclass[12pt,openright,oneside,a4paper,brazil]{abntex2}


\usepackage{amsmath}
\usepackage{amsmath,amsthm,amsfonts,amssymb,amscd}
\usepackage{graphicx}
\usepackage[ruled,vlined]{algorithm2e}
\graphicspath{{./grafos/}{./}} % where to search for the images

%opening
\title{Trabalho de Analise de Algoritmos}
\author{Felipe Augusto Ferreira de Castro}

\begin{document}

\maketitle

\chapter{Coloração de Grafos}

\section{Pseudo-código}

O problema de coloração de grafos é um problema bastante discutido na literatura da área e possui vários algoritmos para soluciona-lo. Portanto, nesta seção apresentaremos o algoritmo usado em nosso trabalho para encontrar a disposição de cores e posteriormente discutiremos a eficiência do algoritmo. A seguir o algoritmo utilizado apresentado em pseudo-código.\linebreak

\begin{algorithm}[H]
\SetAlgoLined
\KwIn{grafo}
\KwOut{Lista com as cores de cada vertice}
declare uma variavel flag;

faça flag = verdadeiro;

atribua a cor 0 para todos os vertices do grafo\;

primeiro vertice do grafo\; 

\While{não atribuido uma cor a todos os vertices}{	
	
	\While{para todos os vertices adjacentes }
	{ 
		olhe a cor do vertice adjacente\;
		
		\eIf{possui a mesma cor que o vertice adjacente}
		{
			
		 faça flag = falso\;
		 	
		 pare o laço\;		
		}{}
	}
	\eIf{flag =  verdadeiro}{
		proximo vertice;
		
	}{  
		faça flag = verdadeiro\;
	    some 1 a cor deste vertice\;
	}
}
\caption{Coloraçao de Grafos}
\end{algorithm}

Explicando de maneira mais informal o algoritmo se resume em alguns passos:
\begin{itemize}
	\item Para cada vertice $v$ do grafo $G$ vamos olhar as cores de seus vertices adjacentes;
	\item vamos avançando da lista de cores ate encontrar uma cor a qual não foi atribuida a nenhum vertice adjacente a $v$;
	\item atribuimos a cor encontrada ao vertice $v$; 
\end{itemize}


\end{document}
